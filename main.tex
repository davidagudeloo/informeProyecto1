\documentclass{article}
\usepackage[utf8]{inputenc}
\usepackage[spanish]{babel}
\usepackage{listings}
\usepackage{graphicx}
\graphicspath{ {images/} }
\usepackage{cite}

\begin{document}

\begin{titlepage}
    \begin{center}
        \vspace*{1cm}
            
        \Huge
        \textbf{Proyecto #1}
            
        \vspace{0.5cm}
        \LARGE
        Parcial 1 - Calistenia (15\%)
            
        \vspace{1.5cm}
            
        \textbf{David Agudelo Ochoa}
            
        \vfill
            
        \vspace{0.8cm}
            
        \Large
        Despartamento de Ingeniería Electrónica y Telecomunicaciones\\
        Universidad de Antioquia\\
        Medellín\\
        Marzo de 2021
            
    \end{center}
\end{titlepage}

\tableofcontents
\newpage
\section{Introducción}\label{intro}
Con el fin de comenzar la preparación para los desafíos a enfrentar en las distintas actividades evaluativas del curso, se propone el ejercicio de generación de un algoritmo, donde se parte de un estado A y se requiere llegar a un estado final B. Además el proceso será documentando mediante herramientas muy poderosas, no solo para el curso, sino también para la vida; tales como Overleaf, Git y Github.

\section{Algoritmo a implementar} \label{contenido}
0.Inicio. \newline 1. Levante la hoja de papel con cuidado de no estropearla. \newline 2. Ubíquela encima de la mesa de modo que no quede sobre las tarjetas. \newline 3. Alinee las tajetas de modo que una de ellas cubra a la otra completamente. \newline 4. Ubique el dedo pulgar en uno de los costados más largos de las tarjetas, de modo que se sienta cómodo. \newline 5. Ubique el dedo índice en el lado siguiente en sentido horario (si no se siente cómodo vuelva al paso 4.). \newline 6. Ubique los tres dedos restantes en el siguiente costado. \newline 7. Agarre ambas tarjetas con las yemas de los dedos de modo que conserve las indicaciones anteriores. \newline 8. Apoye el lado que no está sostenido por sus dedos encima de la hoja. \newline 9. Apoyándose en la tarjeta que está más lejana a la palma de su mano, separelas lentamente procurando conseguir una pirámide. \newline 10. Equilibre las tarjetas de modo que se mantengan sin la ayuda de sus dedos. \newline 11. Si se caen, vuelva al paso 3. \newline 12. Fin :)


\section{Conclusión} \label{conclu}
De esta experiencia he tenido diversos aprendizajes, entre ellos, el no subestimar, pues aunque en un principio la actividad no parecía tener mucha correlación con la programación, es posible obtener varias conclusiones bastante interesantes. Al hacer la comparativa entre el algoritmo que desarrolle en la primera clase puedo evidenciar lo ambiguos que somos a la hora de dar instrucciones, y en contraste con el algoritmo final desarrollado, se identifica que no solo nosotros como programadores solemos ser ambiguos, sino que es extremadamente difícil diseñar un algoritmo para otras personas, pues el lenguaje es tan amplio, que cada quien puede tener una representación o idea de una cosa diferente a los demás. Dandonos cuenta que somos muy afortunados al tener que darle instrucciones a un computador, pues este solo sigue ordenes.


\bibliographystyle{IEEEtran}


\end{document}
